\chapter*{Introduction}
\addcontentsline{toc}{chapter}{Introduction}
\markboth{Introduction}{Introduction}
\label{chap:introduction}
%\minitoc

At this moment, databanks worldwide contain brain images and individual genomes in previously
unimaginable numbers. Combined with developments in data science, these massive data provide
the potential to fully understand the genetic underpinnings of brain disease. However, as it often
happens, society lags behind innovation: different chunks of data, which are stored at different
institutions, cannot always be shared directly due to privacy concerns and legal complexities,
thus, preventing exploitation of big data in the study of brain disorders.\\

In this project we aim at pursuing our work on online learning in distributed medical databases.
We apply a novel computational paradigm for the meta-analysis of large-scale medical datasets
distributed across clinical centers. The rationale of the project is to pool different data parts
without sharing individual information, through advanced multivariate data analysis tool based
on online and distributed learning. Our proposal extends the state-of-art methodology beyond the
simplistic effects of individual genes on individual brain components (mass-univariate analysis),
by analytically exploring how combinations of genes and brain areas interact in concert
(multivariate online-learning). This promising methodology will be applied for the first time
within the large-scale ENIGMA imaging-genetics consortium, providing data of thousands of
subjects from several clinical centers distributed around the world. Thanks to the expertise and
feedback of our UCA excellence partner in biology and clinic (CNRS IPMC, Fondation CHU-
Lenval), the project presents a unique opportunity for the identification and validation of sets of
candidate genetics variants underpinning autism and neuropsychiatric disorders.\footnote{Requeriment document from }


%%% Local Variables: 
%%% mode: latex
%%% TeX-master: "isae-report-template"
%%% End: 
