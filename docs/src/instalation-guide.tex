\chapter{Installation guide}
\section{Requirements}
\subsection{Hardware requirements}
These are the minimum hardware requirements
\begin{itemize}
	\item CPU: 4 cores @ 2 GHz
	\item RAM: 8 GB
	\item Storage: 4 GB
\end{itemize}

\subsection{Software requirements}
\subsubsection{Operating system}
This software runs \textbf{ONLY} on Unix-based OS like\footnote[1]{\textit{\textbf{IMPORTANT: } Windows platforms are not still supported.}}:
\begin{itemize}
	\item Linux: Fedora, CentOS, Debian, Ubuntu, etc.
	\item macOS: El Capitan 10.11 or higher.
\end{itemize}

More information can be found \href{https://success.docker.com/article/compatibility-matrix}{here}.

\subsubsection{Other software}
For running the MetaImaGen software your system needs to have:
\begin{itemize}
	\item Docker version 17 or higher (\href{https://docs.docker.com/install/#supported-platforms}{Installation instructions})
	\item Docker Compose Version 3.0 or higher (\href{https://docs.docker.com/compose/install/}{Installation instructions})
\end{itemize}

\subsection{Getting Started with MetaImaGen}
\begin{enumerate}
	\item Clone the repository by executing: \\
		\codeword{git clone https://github.com/sssilvar/meta-imagen.git}
	\item Run the test program to check if everything is in order:\\
		\codeword{bash meta-imagen/bin/test.sh}
	\item Run the analysis of your images by executing:\\
		\codeword{bash meta-imagen/bin/run_metaimagen.sh [data_folder_path][center_id]}\\
		
		Where \codeword{[data_folder_path]} is the folder containing all the subjects processed by \textbf{FreeSurfer} and a file called \textit{groupfile.csv} that contains the subjects to be processed (IDs) and the diagnostic group where they belong to. An example is shown in table~\ref{table:groupfile}
\end{enumerate}


\begin{table}[!h]
\centering
\begin{tabular}{|l|l|}
\hline
\multicolumn{1}{|c|}{\textbf{sid}} & \multicolumn{1}{c|}{\textbf{dx}} \\ \hline
002\_S\_0954                       & MCIc                             \\ \hline
002\_S\_1070                       & AD                               \\ \hline
$\cdots$                           & $\cdots$                         \\ \hline
\end{tabular}
\caption{\textit{groupfile.csv} structure.}
\label{table:groupfile}
\end{table}